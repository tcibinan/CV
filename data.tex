\newcommand{\cvfirstname}{Andrei}
\newcommand{\cvlastname}{Tcibin}
\newcommand{\cvposition}{Software Engineer}

\newcommand{\cvlocation}{Vilnius, Lithuania, EU}
\newcommand{\cvmail}{tcibinan@gmail.com}
\newcommand{\cvgithub}{@tcibinan}
\newcommand{\cvgithuburl}{https://github.com/tcibinan}
\newcommand{\cvtelegram}{@nameofthelaw}
\newcommand{\cvtelegramurl}{https://t.me/nameofthelaw}

\newcommand{\cvtemplatewhoami}{Who Am I?}
\newcommand{\cvhowami}{
Software engineer with 5+ years experience on Life Sciences and EdTech projects with expertise in cloud technologies (AWS, Google Cloud, Azure), Kubernetes, Elasticsearch and Grid Engine.
Mostly skilled in Java, Spring, Python and Kotlin Multiplatform.
Worked with top pharmaceutical clients.
Strong contributor to open-source projects.
Most of the time I help moving massive on-premises workloads to cloud/hybrid environments.
}

\newcommand{\cvtemplatelanguages}{Languages}
\newcommand{\cvlanguages}{
	\textbf{Russian} - native\\
	\textbf{English} - business\\
	\textbf{German} - elementary\\
	\textbf{Lithuanian} - elementary
}

\newcommand{\cvtemplateexperience}{Experience}
\newcommand{\cvexperience}{
	\begin{entrylist}
        \entry
		{2022 – present}
		{Senior Software Engineer, EPAM Systems}
		{Vilnius, Lithuania, EU}
		{
			I've proceed working on Cloud Pipeline project but also started contributing to other Life Sciences projects bringing more cloud/hybrid solutions to a wider audience.\\
			\texttt{Life Sciences}\slashsep\texttt{Cloud Computing}\slashsep\texttt{Hybrid Computing}}
        \entry
		{2018 – 2022}
		{Software Engineer, EPAM Systems}
		{Saint Petersburg}
		{
			Later on, I started working as a developer on Cloud Pipeline project which is a cloud agnostic solution for a variety of Life Sciences workloads.
			My main focus is on designing and implementing cloud/hybrid adaptive processing solutions for multiple Life Sciences applications which make the most use of cloud technologies with the respect to existing on-premises environments.
			Other contributions include multiple cloud autoscaling solutions and Elasticsearch optimizations.\\
			\texttt{Life Sciences}\slashsep\texttt{Cloud Computing}\slashsep\texttt{Hybrid Computing}}
		\entry
		{2017 – 2018\\\footnotesize{part time}}
		{Junior Software Engineer, EPAM Systems}
		{Saint Petersburg}
		{
			I started my career by contributing to Life Sciences open source projects which are usef for high-throughput sequencing data processing.\\
			\texttt{Life Sciences}\slashsep\texttt{Bioinformatics}}
	\end{entrylist}
}

\newcommand{\cvtemplateeducation}{Education}
\newcommand{\cveducation}{
	\begin{entrylist}
		\entry
		{2014 – 2020}
		{Bachelor/Master's Degrees in Software Engineering, ITMO University}
		{Saint Petersburg}
		{
			I studied Software Engineering and used both my bachelor and master programs to work on my Flaxo project.
		 	Flaxo is an EdTech platform for practical programming courses which automatically tests, analyses code quality and detects plagiarism in programming submissions.\\
			\texttt{EdTech}\slashsep\texttt{Automation}}
	\end{entrylist}
}

\newcommand{\cvtemplateachievement}{Achievements}
\newcommand{\cvachievement}{
	\begin{entrylist}
		\entry
		{2018}
		{Awardee, BioHack}
		{Saint Petersburg}
		{
			During the hackaton our team processed over hundred samples of human virome using multiple bioinformatics tools and post-processed the data to discover how phages and bacteria correlate in viromes.
			I'm proud to say that our team became awardees for the research that we've done.\\
			\texttt{Life Sciences}\slashsep\texttt{Bioinformatics}}
		\entry
		{2017}
		{Finalist, VK Hackaton}
		{Saint Petersburg}
		{
			In two days our team designed and developed a chat bot prototype which helps people find others with similar interests for enjoying offline events together.
			We were competing with a hundred other teams to win the prize and we were lucky enough to be in one of the twenty five finalists.}
	\end{entrylist}
}
