\newcommand{\cvfirstname}{Andrei}
\newcommand{\cvlastname}{Tcibin}
\newcommand{\cvposition}{Software Engineer}

\newcommand{\cvlocation}{Vilnius, Lithuania, EU}
\newcommand{\cvmail}{tcibinan@gmail.com}
\newcommand{\cvgithub}{@tcibinan}
\newcommand{\cvgithuburl}{https://github.com/tcibinan}
\newcommand{\cvtelegram}{@nameofthelaw}
\newcommand{\cvtelegramurl}{https://t.me/nameofthelaw}

\newcommand{\cvtemplatewhoami}{Who Am I?}
\newcommand{\cvhowami}{
Software engineer with 5+ years experience on Life Sciences and EdTech projects with expertise in cloud technologies (AWS, Google Cloud, Azure), Kubernetes, Elasticsearch and Grid Engine.
Mostly skilled in Java, Spring, Python and Kotlin Multiplatform.
Worked with top pharmaceutical clients.
Strong contributor to open-source projects.
Most of the time I help moving massive on-premises workloads to cloud/hybrid environments.
}

\newcommand{\cvtemplatelanguages}{Languages}
\newcommand{\cvlanguages}{
	\textbf{Russian} - native\\
	\textbf{English} - business\\
	\textbf{German} - elementary\\
	\textbf{Lithuanian} - elementary
}

\newcommand{\cvtemplateexperience}{Experience}
\newcommand{\cvexperience}{
	\begin{entrylist}
        \entry
		{2022 – present}
		{Senior Software Engineer, EPAM Systems}
		{Vilnius, Lithuania, EU}
		{
			Nowadays, my main focus is designing and implementing cloud/hybrid solutions for Life Sciences applications which make the most use of cloud technologies with the respect to existing on-premises environments. 
			I contribute to multiple Life Science projects bringing more cloud/hybrid solutions to light.\\
			\texttt{Life Sciences}\slashsep\texttt{Cloud Computing}\slashsep\texttt{Hybrid Computing}}
        \entry
		{2018 – 2022}
		{Software Engineer, EPAM Systems}
		{Saint Petersburg}
		{
			Before that, I contributed a lot to Cloud Pipeline project which is a cloud agnostic solution for a variety of Life Sciences workloads.
			My contributions include implementing cloud/hybrid processing solutions, multiple autoscaling solutions and Elasticsearch optimizations.\\
			\texttt{Life Sciences}\slashsep\texttt{Cloud Computing}\slashsep\texttt{Hybrid Computing}}
		\entry
		{2017 – 2018\\\footnotesize{part time}}
		{Junior Software Engineer, EPAM Systems}
		{Saint Petersburg}
		{
			Earlier, I contributed to Life Sciences open source projects which are used for high-throughput sequencing data processing.\\
			\texttt{Life Sciences}\slashsep\texttt{Bioinformatics}}
	\end{entrylist}
}

\newcommand{\cvtemplateeducation}{Education}
\newcommand{\cveducation}{
	\begin{entrylist}
		\entry
		{2014 – 2020}
		{Bachelor/Master's Degrees in Software Engineering, ITMO University}
		{Saint Petersburg}
		{
			Before that, I studied Software Engineering and used both my bachelor and master programs to work on my Flaxo project 
			which is an EdTech platform for practical programming courses with automated testing, code quality analysis and plagiarism detection.\\
			\texttt{EdTech}\slashsep\texttt{Automation}}
	\end{entrylist}
}

\newcommand{\cvtemplateachievement}{Achievements}
\newcommand{\cvachievement}{
	\begin{entrylist}
		\entry
		{2018}
		{Awardee, BioHack}
		{Saint Petersburg}
		{
			My team and I processed over a hundred samples of human viromes using multiple bioinformatics tools and post-processed the data to discover how phages and bacteria correlate.\\
			\texttt{Life Sciences}\slashsep\texttt{Bioinformatics}}
		\entry
		{2017}
		{Finalist, VK Hackaton}
		{Saint Petersburg}
		{
			Our team competed with a hundred other teams to design and develop a chat bot prototype in just two days which helps people find others with similar interests to enjoy offline events together.}
	\end{entrylist}
}
